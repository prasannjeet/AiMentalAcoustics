\chapter*{\rm\bfseries Abstract}
\label{ch:abstraten}

In contemporary society, stress has emerged as a prevalent mental health concern with significant implications for overall well-being. Traditional methods of stress assessment, often invasive or reliant on self-reporting, have limitations in terms of efficiency and potential biases. This study introduces the AiMentalAcoustics system, an innovative approach that harnesses the power of Artificial Intelligence (AI) to detect and predict stress levels through non-invasive voice analysis. By analyzing various vocal attributes, such as pitch, intensity, and frequency, the system aims to extract meaningful insights about an individual's emotional and mental state.

Initially, the research delved into the correlation between EEG (electroencephalogram) data and voice acoustics to pinpoint key features indicative of stress. Leveraging these insights, an AI model was developed based purely on voice data, offering a convenient means for stress detection. The AiMentalAcoustics system not only promises rapid and non-intrusive stress detection but also paves the way for timely interventions, improved mental health management, and enhanced well-being outcomes. By juxtaposing voice acoustics with EEG data analysis, the study endeavors to establish an efficient stress detection paradigm that could revolutionize the way we perceive and address mental health in the modern age.

