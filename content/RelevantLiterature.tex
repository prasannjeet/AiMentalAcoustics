\chapter*{\rm\bfseries Literature}
\label{ch:literature}

\mcgillguidelines: The comprehensive review of the literature must sufficiently demonstrate the student’s knowledge of and expertise in their research areas and should be broad enough to apply to each research question in the thesis. The review of the literature can additionally include various types of content, such as:

\begin{itemize}
    \item{A review providing a reader who is relatively less familiar with the research topic (e.g., an internal/external member of an oral defence committee with adjacent but not direct expertise) an introduction to the general domain.}
    \item{An explanation of the overall rationale for how and why the subsequent studies were conducted. For example, the literature underlying the research questions must be sufficiently discussed.}
    \item{A review of fundamental theories underlying the subsequently presented work, or to explain why certain approaches were not taken in the study(ies) presented.}
\end{itemize}

The literature review must be in line with disciplinary expectations. The review can be incorporated in the Introduction chapter, addressed in a standalone chapter, or distributed across multiple chapters.